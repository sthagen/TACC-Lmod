\documentclass{beamer}

% You can also use a 16:9 aspect ratio:
%\documentclass[aspectratio=169]{beamer}
\usetheme{TACC16}

% It's possible to move the footer to the right:
%\usetheme[rightfooter]{TACC16}

%% page 
%\begin{frame}{}
%  \begin{itemize}
%    \item
%  \end{itemize}
%\end{frame}
%
%% page 
%\begin{frame}[fragile]
%    \frametitle{}
% {\tiny
%    \begin{semiverbatim}
%    \end{semiverbatim}
%}
%  \begin{itemize}
%    \item
%  \end{itemize}
%
%\end{frame}

\begin{document}
\title[Lmod]{How Lmod processes TCL modulefiles}
\author{Robert McLay} 
\date{November 8, 2022}

% page 1
\frame{\titlepage} 


% page 2
\begin{frame}{Outline}
  \center{\includegraphics[width=.9\textwidth]{Lmod-4color@2x.png}}
  \begin{itemize}
    \item Original Idea was just to support Lua based modules
    \item Hard to get sites to translate their existing modules
      (including TACC!)
    \item So how to support TCL modulefiles?
    \item What strategies are possible?
    \item What technique does Lmod use
    \item Why it is never going to be perfect.
  \end{itemize}
\end{frame}

% page 3
\begin{frame}{How could TCL modulefiles be supported?}
  \begin{itemize}
    \item Could write a TCL interpreter in Lua
    \item Ugh!, too much work, Hard to maintain, Hard to get right
    \item Could rewrite much of the Lmod lua code in TCL
    \item Ugh! (same reasons!)
    \item Try something else
  \end{itemize}
\end{frame}

% page 4
\begin{frame}{Something good enough}
  \begin{itemize}
    \item I tried something that I could do
    \item Create (really steal) a pure TCL module interpreter
    \item Convert it to work for Lmod.
    \item Called tcl2lua.tcl
  \end{itemize}
\end{frame}


% page 5
\begin{frame}{Where tcl2lua.tcl cam from}
  \begin{itemize}
    \item It was the freely available pure TCL env. module code
    \item This was before Xavier took over Tmod
    \item It was simple enough to understand and convert.
  \end{itemize}
\end{frame}

% page 6
\begin{frame}{TCL is one of my least favorite languages}
  \begin{itemize}
    \item Its parser is very line-oriented
    \item So it is very picky about what goes where
    \item My intuition about the language is almost always wrong.
  \end{itemize}
\end{frame}

% page 7
\begin{frame}{Surprising help from stackoverflow.com}
  \begin{itemize}
    \item I have had many questions about how TCL works
    \item I am always shocked that my TCL question would get answered
    \item Grateful, but amazed!
  \end{itemize}
\end{frame}

% page 8
\begin{frame}{Questions for stackoverflow.com}
  \begin{itemize}
    \item How does the puts command work?
    \item How break works?
    \item How the child interpreter works
  \end{itemize}
\end{frame}

% page 9
\begin{frame}{How tcl2lua.tcl works}
  \begin{itemize}
    \item Let the TCL interpreter evaluate regular TCL commands
    \item Convert the TCL setenv, prepend-path etc
    \item $\Rightarrow$ into Lua setenv(), prepend\_path() strings
    \item This lua output is then interpreted by Lmod as a Lua module
    \item This works very well most of the time
    \item However ...
  \end{itemize}
\end{frame}

% page 10
\begin{frame}[fragile]
    \frametitle{When things go awry}
  \begin{itemize}
    \item Suppose you have TCL modules \textbf{Centos} and \textbf{B}
  \end{itemize}
 {\tiny
    \begin{semiverbatim}
Centos::

    #%Module
    setenv SYSTEM\_NAME Centos

And B::

    #%Module
    module load Centos

    if { $env(SYSTEM\_NAME) == "Centos" } {
       # do something
    }

    \end{semiverbatim}
}
\end{frame}

% page 11
\begin{frame}[fragile]
    \frametitle{Converting the TCL \textbf{B} into Lua}
 {\tiny
    \begin{semiverbatim}
   load("Centos")
   LmodError("can't read \"env(SYSTEM\_NAME)\": no such variable")
    \end{semiverbatim}
}
  \begin{itemize}
    \item Trouble: the TCL \textbf{load} command $\Rightarrow$
      \texttt{load("Centos")}
    \item Cannot get the TCL load command to be evaluated before the
      TCL if block
  \end{itemize}

\end{frame}


% page 12
\begin{frame}{The best solutions would be: }
  \begin{itemize}
    \item In this case use TCL code to get the name of the Linux OS 
    \item Or use lsb\_release inside the \textbf{B} modulefile
    \item Or translate the TCL \texttt{B} module into Lua
    \item Note that you can have both a TCL B/1.0 module \textbf{and}
      a B/1.0.lua module in the same directory
    \item Lmod will always chose the Lua modulefile over the TCL one.
    \item Tmod will ignore the *.lua file (no \#\%Module start line)
  \end{itemize}
\end{frame}

% page 13
\begin{frame}[fragile]
    \frametitle{The \textbf{B} translated into Lua}
 {\tiny
    \begin{semiverbatim}
   load("Centos")
   if (os.getenv("SYSTEM\_NAME") == "Centos") then
     -- Do something
   end
    \end{semiverbatim}
}
  \begin{itemize}
    \item Note that the Centos module DO NOT need to be translated
      into Lua.
  \end{itemize}

\end{frame}

% page 15
\begin{frame}{Next time}
  \begin{itemize}
    \item Lmod 8.0+ brought many improvements to TCL module support
    \item TCL break command
    \item Optional Integration of TCL interpreter into Lmod (saves
      time)
    \item Support for is-loaded 
    \item Support for is-avail and why I resisted supporting that for
      TCL modulefiles for so long
    \item Special features of setenv and pushenv in TCL
    \item How tcl2lua.tcl supports the \textbf{puts} command
    \item New bugs found when integrating the TCL interpreter into Lmod
  \end{itemize}
\end{frame}



% page 25
\begin{frame}{Future Topics}
  \begin{itemize}
    \item Next Meeting: December 6th 9:30 US Central (15:30 UTC)
  \end{itemize}
\end{frame}

\end{document}
